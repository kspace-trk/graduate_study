\chapter{はじめに}
個別指導塾では,生徒が学校の進度に追いつくことを目的として,夏期講習や冬期講習などの長期講習が実施される.長期講習に先立ち,生徒は塾から提案されたカリキュラムをもとに受講科目とコマ数,および受講可能日時を申し出る.一方,講師は担当可能科目と勤務可能日時を提出する.時間割は,生徒が希望する科目とコマ数,および受講可能な日時に加え,講師の担当可能科目と勤務可能日時に沿ったものでなければならない.また,生徒と講師の満足度を高めるために,生徒にとってコミュニケーションのとりやすい講師が担当になっていることが望ましい.時間割を手作業で作成する場合には,半月の講習の時間割作成時間が約70時間に達し,時間割作成者の業務負担が非常に大き
いという事例も報告されている.

本研究では,時間割作成者の業務負担を減らすことを目的とし,個別指導塾の長期講習における時間割作成支援システムを構築する.
% 参考文献はこんな感じで引用します.
% \cite{Quinlan96}
% \cite{Quinlan89}
% \cite{Quinlan93}
% \cite{Iba94}
% \cite{C50}
% \cite{Nakayama00}
% \cite{Otani06}
% \cite{Otani04}
% \cite{Saino88}
% \cite{Ishibashi11}
% \cite{Ushijima11}


% 図はこんな風に入れます.

% \begin{figure}[tbhp]
% \begin{center}
% \includegraphics[scale=0.95]{image/se.eps}
% \caption{共生進化における2つの集団}
% \label{fig:02se}
% \end{center}
% \end{figure}

% 図番号は図\ref{fig:02se}のように参照します.
