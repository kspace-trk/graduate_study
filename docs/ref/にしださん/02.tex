\chapter{既存の時間割作成支援システム}
% 本章では,個別指導塾における時間割作成支援システムの先行研究と問題点について述べる.

黒沢[黒沢10]は個別指導塾の通常講習において,Excelをインタフェースとした時間割作成支援システムを開発している.システムは,生徒の氏名,学年,希望科目,希望曜日と講師の得意科目,不得意科目および講師の勤務可能日時が入力されると,各生徒が受講する各授業の担当講師を決定するための表をExcel形式で出力する.時間割作成者は,出力された表から生徒の満足度が高くなるような担当講師を選択する.このとき,選択された講師の得意科目,不得意科目と講師が1日に担当する生徒数から算出した生徒の満足度が	×,△,〇,◎の4段階で表示される.時間割作成者は,生徒の満足度を参照しながら担当講師を選択することができるため,時間割作成業務の負担を減らすことができる.

黒沢のシステムは通常講習での利用のみ対応しており,長期講習での利用は想定されていない.また,生徒の満足度と講師の得意科目,不得意科目を考慮することはできるが,生徒の受講可能日時や講師の満足度,生徒と講師の相性を考慮することはできない.さらに,時間割作成者が手動で担当講師を割り当てる必要がある.したがって,長期講習での利用を可能とし,より時間割作成業務の負担を減らすためには,任意の講習期間で,生徒の受講可能日時と講師の勤務可能日時を考慮し,生徒と講師の満足度を高め,自動で担当講師を割り当てるシステムを構築する必要がある.