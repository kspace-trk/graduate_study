\chapter{評価実験}

\section{実験手法}
スクールIEの時間割作成者A,Bを被験者として評価実験を実施した.本システムを3回実行し,担当割り振りRC1からRC3を用意した.ただし,$w$は0.5,交叉手法は一様交叉,選択手法はトーナメント選択とする.加えて,過去の冬期講習における担当割り振りRC4を用意した.各RCを比較させ,以下の質問について1〜5の5段階で,5に近づくにつれ評価が高くなるように評価させた.また,各RCに対して感じたことや意見を自由記述として集めた.調査にはGoogle フォームを用いた.アンケート画面を図\ref{A1}〜\ref{A3}に示す.

\def\MARU#1{{\rm\ooalign{\hfil\lower.168ex\hbox{#1}\hfil \crcr\mathhexbox20D}}}
\begin{enumerate}
  \item[\MARU{1}] RC1〜RC3をもとに時間割を作成することは可能か.
  \item[\MARU{2}] RC1〜RC3をもとに時間割を作成した場合,業務時間を減らすことができるか.
\end{enumerate}

さらに,本システムを使用させ,以下の質問につい1〜5の5段階で,5に近づくにつれ評価値が高くなるように評価させた.また.本システムに対して感じたことや意見を自由記述として集めた.調査にはGoogle フォームを用いた.アンケート画面を図\ref{A4}〜\ref{A5}に示す.

\begin{itemize}
  \item[\MARU{3}] システムは使いやすいか.
  \item[\MARU{4}] デザインは適切なものか.
  \item[\MARU{5}] 本システムは実際の業務で使用可能か.
  \item[\MARU{6}] 本システムを使用することによって業務時間を減らすことは可能か.
\end{itemize}

\section{実験結果}
質問\MARU{1},質問\MARU{2}について被験者の回答を表\ref{table5.1},質問\MARU{3}から質問\MARU{6}について被験者の回答を表\ref{table5.2}に示す.被験者の自由記述を付録Bに示す.


\begin{table}[htbp]
\begin{center}

  \caption{被験者の各RCに対する評価}
  \label{table5.1}
  \begin{tabular}{|c|c|c|c|c|c|c|} \hline
        & \multicolumn{2}{|c|}{RC1}       & \multicolumn{2}{|c|}{RC2} & \multicolumn{2}{|c|}{RC3}\\\hline
  質問  & \MARU{1} & \MARU{2}             &  \MARU{1} & \MARU{2}      & \MARU{1}       & \MARU{2} \\\hline
  A     & 4        & 4        & 4         & 5                         & 4              & 5        \\\hline
  B     & 5        & 5        & 5         & 5                         & 5              & 5        \\\hline
  \end{tabular}

\end{center}
\end{table}

\begin{table}[htbp]
\begin{center}

  \caption{被験者の本システムに対する評価}
  \label{table5.2}
  \begin{tabular}{|c|c|c|c|c|} \hline
  質問 & \MARU{3} & \MARU{4} & \MARU{5} & \MARU{6}\\\hline
  A    & 4        & 3        & 4        & 4       \\\hline
  B    & 3        & 4        & 4        & 5       \\\hline
  \end{tabular}
\end{center}
\end{table}

本システムが生成した担当割り振りと本システムに関して肯定的な評価を得ることができた.しかし,付録Bでは,「講師ごとに意図的に勤務率を変動できる設定にできたら,より良くなります.」,「長期講習中で,長期間に渡って生徒と講師の日程が合うようにしたいしたい.」,「 学年ごとの講師が担当可能な科目を考慮したい.」という要望があった.