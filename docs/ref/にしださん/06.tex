\chapter{考察}
本システムで決定したRCに対しての評価はどれも4以上であることから,本システムの担当割り振り決定手法の有用性が示されたといえる.特に,被験者からの自由記述では,「おそらく時間割作成業務は半分以下になると考えます.」という意見があった.被験者は,時間割作成業務において,講師の勤務率や,生徒の受講可能日時と講師の勤務可能日時の合致度を考慮することが重要だと認識しているためだと考えられる.しかし,「講師ごとに意図的に勤務率を変動できる設定にできたら,より良くなります.」という意見や,「長期講習中で,長期間に渡って生徒と講師の日程が合うようにしたい.」という意見,「学年ごとの講師が担当可能な科目を考慮したい.」という意見が挙げられた.また,通常授業において生徒を担当する講師が「もともと担当であるかどうかも重要」という意見があった.生徒と講師にとって,元々面識があり話しやすい関係であるほど,円滑な授業を実施することができるため,両者の満足度が高くなるためだと考えられる.

より時間割作成者の要望に沿った担当割り振りを決定するためには,講師ごとの理想勤務率を決定する機能の導入や,生徒の受講可能日時と講師の勤務可能日時の合致度を日単位,週単位で評価する機能の導入が必要である.また,「新講習ファイルへの入力作業が大変そう」という意見から,受講科目とコマ数.受講可能日時,勤務可能日時をスマートフォンやPCを用いて送信可能な仕組みを構築することで,より有用性が増し評価が上がると考えられる.さらに,通常授業や過去の長期講習の担当割り振りを考慮することによって,生徒と講師にとってより満足度が高い担当割り振りを決定することが可能だと考えられる.