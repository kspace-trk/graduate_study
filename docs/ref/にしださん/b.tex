\chapter{評価実験における被験者の自由記述}
評価実験における各RCとシステムについての自由記述結果を以下に示す.


\textgt{各RCについての自由記述}


【RCを比較して良かった点】
\begin{itemize}
  \item 講師と生徒の相性がとても大切だと考えている.
  \item スケジュール作成を行う前の準備が極めて重要なため,事前に何らかの計算予測があると作業効率が格段に上がると考えます.
  \item 全ての講師に均等に割り振れたこと自体素晴らしいです.
  \item おそらく時間割作成業務は半分以下になると考えます.これまで,担当講師決めやシミュレーションで5時間,スケジュール作成で50時間以上かけていました.
  \item 生徒と講師の希望日時の整合を図ることは人の手では膨大な時間を要するため,業務時間を軽減できる.
\end{itemize}

【RCを比較して不満な点,気になる点】
\begin{itemize}
  \item 準備が大変そう
  \item 講師ごとに意図的に勤務率を変動できる設定にできたら,より良くなります.お金を稼ぎたい講師,あまり入りたくない講師,退職予定の講師,勤務したての講師がいるため.
  \item 実際の時間割作成業務で実行できたらより効果がわかる.
  \item 現段階はあくまで,時間一致かつ勤務コマ数の均等化であるため,他の複数要素が計算に入れられるかどうかが実用化のポイントとなります.時間や勤務割合以外に必要な要素
  \begin{enumerate}
    \item 学年ごとの講師が担当可能な科目を考慮したい.
    \item 長期講習中で,長期間に渡って生徒と講師の日程が合うようにしたい.
    \item 生徒と講師の相性(もともと担当であるかどうかも重要)*授業を提供することが第1位にくるため,特に1,2は欠かせない.
    \item 生徒と生徒の相性(1対2の授業の場合)
    \item 講師毎の勤務バランス(意図的に未来の主軸講師に多く勤務させたいことあり)
    *他複数要素はありますが,際限がなくなるため,実際はそこまで意識できていないです.
  \end{enumerate}
\end{itemize}

\textgt{システムについての自由記述}


【本システムを使用して良かった点】
\begin{itemize}
  \item どこで何ができるのかすぐにわかる.
  \item 業務時間を短縮できると感じた.
  \item ゲーム感覚で作業できるので良いと思いました.
  \item ゲームのような雰囲気で良い.楽しみながら作業できる.
  \item 重い作業なので,楽しみながらできる.
\end{itemize}

【本システムを使用して不満な点,気になる点】
\begin{itemize}
  \item Excelとシステムを行き来するのは少し大変
  \item 実際に運用したい.
  \item データ入力をより迅速にできるような機能があってもいいと思う.
  \item 誰にでも使えるような仕様にするべき.
  \item 運用するまでの完成度を高められたら良い.
  \item デザインの配色が赤と黒なので,少し目が疲れました.
\end{itemize}