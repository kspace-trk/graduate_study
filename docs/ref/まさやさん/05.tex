\chapter{評価実験}
本章では,構築したシステムの有用性を示すために行った評価実験の内容と結果について説明する.

\section{実験内容}
トランプに詳しい,マジシャン,カーディスト,コレクター合計6名に加え,デザイナー2名,一般学生9名,合計17名を被験者としてアンケート,ヒアリングによる評価実験を行った.
被験者に本システムを利用させた後,以下の項目について5に近いほど高評価とする1~5の5段階で評価させた.

\begin{itemize}
    \item これからトランプのデザインを作りたいと思っていますか。
    \item トランプに限らずデザインは得意ですか。
    \item 評価の繰り返しでデザインはあなたの好みに近づいていきましたか? 
    \item 本システムで生成されたデザインの満足度はどのくらいでしたか? 
    \item 本システムが生成した、あなたの評価が高かったデザインのトランプはマジックに適していると思いますか。
    \item 本システムが生成した、あなたの評価が高かったデザインのトランプはカーディストリーに適していると思いますか。
    \item 新しいトランプのデザインを作る際に本システムで生成されたデザインからヒントを得ることができると思いましたか? 
\end{itemize}

また,各項目の評価理由,システム全体に関する要望などに関して自由記述により調査した.
アンケート調査ではGoogleフォームを用いた.
加えて,生成した中で最も評価の高かったデザインを3枚まで提出させた.
本実験で用いたGoogleフォームの画面を付録B,提出されたデザインを付録Cに示す.


\section{実験結果}
トランプに知見があるマジシャン,カーディスト,コレクターを被験者区分Aとし,それ以外を被験者区分Bとする.
表\ref{result}に本システムの有用性と満足度に対する平均評価値を示す.
すべての項目で平均評価値が中央値を超えており,本システムの有用性が示せたといえる.
表\ref{result01}に区分Aの被験者のデザイン生成の意欲度を示す.
「まったく作りたくない」「作りたくない」と回答した人はおらず,区分A全員がデザイン製作に対して意欲があることが示された.
表\ref{result02}に項目別のデザイン適正度を示す.
本システムでは,マジックおよびカーディストリーに使うトランプのバックデザインを生成することを想定しており,それぞれの平均中央値が中央値の3を超えているため,使用目的別にデザインをみても本システムが有用であることが示された.
表\ref{result03}に本システムで生成したデザインに対する購入意欲を示す.
「購入したい」と回答した人が5.9%であり,デザインの満足度の平均評価値と比べても少ない割合である.
ダウンロード機能の必要性を表\ref{resultlast}に示す.
被験者全員が「必要」「どちらかというと必要」と回答しており,ダウンロード機能の必要性が示された.

各項目に対する自由記述およびヒヤリングの結果を表\ref{free}~\ref{free7}に示す.

\begin{table}[htbp]
	\centering
	\caption{本システムの有用性と満足度に対する平均評価値}
	\begin{tabular}{|l|c|c|c|} \hline
    \multicolumn{1}{|c|}{質問項目} & \multicolumn{3}{|c|}{平均評価値} \\ \cline{2-4}
                                &全体 &区分A & 区分B\\ \hline
	デザイン生成時の好みの反映度合い & 4.47 & 4.17  & 4.63\\ \hline
    生成されたデザインの満足度 & 4.24 & 3.83 & 4.45 \\ \hline
    生成されたデザインからヒントやアイデアの取得の可否 & 3.82 & 4.50 & 3.63 \\ \hline
	システムの使いやすさ & 3.64 & 4.16  &3.36 \\ \hline 
	\end{tabular}
	\label{result}
\end{table}

\begin{table}[htbp]
    	\centering
    	\caption{区分Aの被験者のデザイン生成の意欲度}
    	\begin{tabular}{|l|r|} \hline
    	\multicolumn{1}{|c|}{デザインを作りたいか} & \multicolumn{1}{|c|}{割合}   \\ \hline
        とても作りたい & 66.6\%   \\ \hline
        まあまあ作りたい & 16.7\%  \\ \hline
        どちらでもない  & 16.7\%   \\ \hline
        作りたくない& 0\%   \\ \hline
        まったく作りたくない  & 0\%   \\ \hline 
    	\end{tabular}
    	\label{result01}
\end{table}

% \begin{table}[]
%     \centering
%     \caption{デザインに対する得意不得意の割合}
%     \begin{tabular}{|l|r|}
%     \hline
%     \multicolumn{1}{|c|}{回答} & \multicolumn{1}{c|}{割合} \\ \hline
%     得意                       & 0%                      \\ \hline
%     どちらかというと得意               & 17.6%                   \\ \hline
%     どちらでもない                  & 17.6%                   \\ \hline
%     どちらかというと不得意              & 41.2%                   \\ \hline
%     不得意                      & 23.5%                   \\ \hline
%     \end{tabular}
% \end{table}

\begin{table}[htbp]
	\centering
	\caption{項目別のデザイン適正の平均評価値}
	\begin{tabular}{|l|c|c|c|} \hline
    \multicolumn{1}{|c|}{質問項目} & \multicolumn{3}{|c|}{平均評価値} \\ \cline{2-4}
                                &全体 &区分A & 区分B\\ \hline
                                マジックに適しているか & 3.65 & 4.33  & 3.45\\ \hline
                                カーディストリーに適しているか  & 3.47 & 3.67 & 3.36 \\ \hline
	\end{tabular}
	\label{result02}
\end{table}

\begin{table}[htbp]
        \centering
        	\caption{生成したデザインに対する購入意欲}
        \begin{tabular}{|l|r|}
        \hline
        \multicolumn{1}{|c|}{回答} & \multicolumn{1}{c|}{割合} \\ \hline
        購入したい                    & 5.9%                    \\ \hline
        どちらかというと購入したい            & 64.7%                   \\ \hline
        どちらかというと購入したくない          & 23.5%                   \\ \hline
        購入したくない                  & 5.9%                    \\ \hline
        \end{tabular} 
        \label{result03}
\end{table}



\begin{table}[htbp]
    \centering
    \caption{ダウンロード機能の必要性}
    \begin{tabular}{|l|r|} \hline
    \multicolumn{1}{|c|}{回答} & \multicolumn{1}{|c|}{割合}   \\ \hline
    必要 & 82.4\%   \\ \hline
    どちらかというと必要 & 17.6\%  \\ \hline
    どちらかというと必要ない  & 0\%   \\ \hline
    必要ない& 0\%   \\ \hline
    \end{tabular}
    \label{resultlast}
\end{table}

\begin{table}[htbp]
    \centering    \caption{デザインの満足度に対する回答結果}
    \begin{tabular}{|c|p{7em}|p{26em}|} \hline
        被験者区分 & \multicolumn{1}{|c|}{回答} & \multicolumn{1}{|c|}{自由記述} \\ \hline
         & とても満足 & シンプルめで、かつ洗練された幾何学模様をあしらったデザインがいい というニーズがあり、それを満たすデザインが完成されたから。\\ \cline{2-3}
          &  満足& あまり見ない組み合わせのデザインができたため \\\cline{3-3}
        A  & & 自分では思いつかないようなデザインが出てきた。\\\cline{3-3}
          & & 即興で多数のデザインができたため \\ \cline{2-3}
          & どちらでもない & 良さそうなデザインも出来ることがありますが最初に選択したデザインの影響が大きすぎる気がします。\\ \cline{2-3}
          & 満足していない& 初期に選択するデックが少なくデザイン生成を繰り返しても好みに近ずかなった \\ \hline
        & とても満足 & 初回でディテールを決めて、2回目〜3回目で詳細を詰めていくような感覚で、直感的に自分の好みへ近づけることができたから \\\cline{3-3}
         & &   デザインの知識がなくてもきれいなデザインを生成できる \\\cline{3-3}
         & & 毎回違うものが生成され、より緻密なデザインが生成されていったところ \\\cline{3-3}
         & & 自分の好きなデザインに近づいていく実感が強く、好みのデザインに仕上げやすい\\\cline{3-3}
        B & & 直感とクリック操作のみでデザインができる点 \\\cline{3-3}
         & &良いと思った直線や円が次の世代に活かされていた\\\cline{2-3}
         & 満足& 色を自分で決定できるので、作りたいデザインがある程度明確になり、それに沿って作っていけたから \\\cline{3-3}
         &  &図形の配置がいい感じのところにきた\\\cline{3-3}
         & &自分の好きじゃないデザインが減っていき,自分の好みに近いデザインが増えたから \\ \cline{2-3}
         & どちらでもない&途中から全く同じデザインのカードが並ぶようになってきてしまった \\ \hline

    \end{tabular}
    \label{free}
\end{table}

\begin{table}[htbp]
    \centering
    \caption{新しいデザインのヒント,アイデア取得の可否に対する回答結果}
    \begin{tabular}{|c|p{7em}|p{26em}|} \hline
        被験者区分 & \multicolumn{1}{|c|}{回答} & \multicolumn{1}{|c|}{自由記述} \\ \hline
        & とてもヒントを &奇想天外な思いもつかないデザインを生成してくれるため。\\\cline{3-3} 
         & 得ることができ&自分では思いつかなそうなデザインだったので! \\ \cline{3-3}
         & た&アイデアは無いが好みはある。という場合でも、このようなシステムを活用すると「好み」から「アイデア」を生み出すことができるから。\\\cline{3-3}
        A & & 複数のトランプがあって、これをくっつけるとどうなるんだ??と日頃思っていたから。\\\cline{2-3}
         & ヒントを得ることができた&このデザインをもとにデックを作りたいなと思ったため \\ \cline{2-3}
         &ほとんどヒントを得ることがで &元々あるデザインから要素をとってきていると思うのでそれだったら普通のデザインを見ながらでもできるのではないか。\\\cline{3-3} 
         &きなかった & しかし、たまに予期していないデザインもあった。ただそれはデザインの整合性が取れていないように感じた \\ \hline
        & とてもヒントを得ることができた& 変形した四角形はデザインに組み込みづらいので、可視化されるとイメージしやすい\\\cline{2-3}
         & ヒントを得ることができた& 同じ図形でも少し位置が変わるだけで印象が変わったりするが、微調整を人間がやると面倒くさい。それを機械が高速かつ大量に提案してくれるので、負担が少ないし印象の違いも発見しやすくなっていると思ったから\\ \cline{2-3}
         & 少しだけヒント & これとこれを組み合わせたいなと感じたシーンがあったため \\\cline{3-3}
        B & を得ることができた&デザイン性がわからなくても面白いデザインが生成ができたから\\\cline{2-3}
         & ヒントを得ることができなかった& 難しい \\ \cline{2-3}
         & 全くヒントを得&作成しないので \\ \cline{3-3}
         & ることができなかった& 基本的にトランプのデザイン性は重視しないため\\\hline

    \end{tabular}
    \label{free2}
\end{table}

\begin{table}[htbp]
    \centering
    \caption{トランプを作る際にシステムを使いたいかに対する回答結果}
    \begin{tabular}{|c|p{7em}|p{26em}|} \hline
        被験者区分 & \multicolumn{1}{|c|}{回答} & \multicolumn{1}{|c|}{自由記述} \\ \hline
        & 使いたい&初めてデックを作りたいがいまいちデザインが浮かばない人には良いと思う\\ \cline{3-3}
         & &便利なので\\ \cline{3-3}
        A & &ヒントを得るという点で有効だから\\ \cline{3-3}
         & &自分の腕ではデザインしきれないと思ったから\\\cline{3-3}
         & &0から自分の好みを具現化することは自分には困難だから。\\ \cline{3-3}
         & &複数の案を簡単に作れるのは魅力だと感じました\\ \hline
        & 使いたい&とても便利だと思うし変わりがない\\ \cline{3-3}
         & & デザイナー的な知見のない素人にとっては、お手軽にそれっぽいデザインを作れるということが非常にありがたいと感じたため\\ \cline{3-3}
        B & &ヒントを得るという点で有効だから\\\cline{3-3}
         & &考える候補を増やしてくれるから\\\cline{3-3}
         & &実際のトランプのデザインから新しいものができるので、手軽で実用的\\\hline

    \end{tabular}
    \label{free3}
\end{table}

\begin{table}[htbp]
    \centering
    \caption{システムの使いやすさに対する回答結果}
    \begin{tabular}{|c|p{7em}|p{26em}|} \hline
        被験者区分 & \multicolumn{1}{|c|}{回答} & \multicolumn{1}{|c|}{自由記述} \\ \hline
         & とても使いやす&わかりやすい \\\cline{3-3}
        A&い &柄の好みも直感的に選択でき、好きor嫌いの2択ではなく、1~5段階で評価できることにより、自分の好みをデザインとして具現化できる可能性が高いと感じたから。\\ \cline{2-3}
        & まあまあ使いやすい & とても簡単な操作で作ったり試せるので使いやすく誰でも気軽に使えると思いました\\\cline{3-3}
         & & 回数が多少面倒くさいところがある\\ \hline
         & まあまあ使いやすい&操作が直感的で使いやすかった\\\cline{3-3}
         &すい &UIはあまり考えられていなかったがシステムのレスポンスやビジュアルはとても良くてやっていて楽しかった\\ \cline{2-3} 
         & どちらともいえ&最終的にデザインを決定したとき、どのデザインが残るのかが分かりにくかったため。\\\cline{3-3}
         &ない &キープされているデザインは枠の色を変えるなど、わかりやすくする工夫があればさらに良いと感じた。\\\cline{3-3}
         B& & 色変更や各種ボタンは、可能であればデザインの選択カラムと分割し、スクロールに追従してくれるような仕組みだと嬉しいと思った。\\\cline{3-3}
         & & キープが2枚まで、などの仕様がわかりにくかったため、システム内にもう少し使い方の説明文があると良いと感じた。\\\cline{3-3}
         & & 一つひとつのデザインに対して、スライダーを調整するのはやや疲れる。\\\cline{3-3}
         & &システムの説明がPDFなのは、使う側として少し使いづらいと感じた。\\ \hline
        
        
        

    \end{tabular}
    \label{free4}
\end{table}

\begin{table}[htbp]
    \centering
    \caption{システムの使いやすさに対する回答結果2}
    \begin{tabular}{|c|p{7em}|p{26em}|} \hline
        被験者区分 & \multicolumn{1}{|c|}{回答} & \multicolumn{1}{|c|}{自由記述} \\ \hline
        
         & どちらともいえない& デザインを保存した際に保存済みかどうかがわからなかった。\\\cline{3-3}
         & & 生成されたデザインの数が多く評価が疲れる\\\cline{3-3}
          & & 評価値を入力するバーと上下のデザインの距離が近いため、評価値を入力するときに少しやりづらいと感じた。\\ \cline{2-3}
         &少し使いづらい & デザイン生成過程のボタンは一番下までスクロールしないと押せないのは少し不便に感じた(特に1回で表示されるデザイン候補が多いので)\\\cline{3-3}
         & & 操作ボタンが全て同じデザインで文字のみで判別するので、アイコンもしくは色が変わっているとより分かりやすいと感じた。\\\cline{3-3}
        B & & 途中で好きな色に変更できるのはとても良いと感じたが、色を決定しないとデザインに反映されないのは、色の微調整が大変。\\\cline{3-3}
         & & 画面右端に生成回数が表示されているが、気づきにくいと感じた。\\\cline{3-3}
         & & 候補に出てくる同じデザインを排除して候補数を減らすとユーザの負担軽減になると感じた。\\\cline{3-3}
         & & 線の色と背景色を好きな色に変更できるのは良いと感じた。\\\cline{3-3}
         & &再生成機能が便利。\\\cline{3-3}
         & &ダウンロード機能がありがたい。\\\cline{3-3}
         & & 既存のデザインから候補をつくっているので、初期集団からある程度良いデザインが出てきた。\\\cline{3-3}
         & & デザインを保存できる機能が良いと感じた。\\\cline{3-3}
         & & デザイン候補がしっかりしていた。\\\cline{3-3}
         & & 次のデザインを押下したときの画面の切り替わりがわかりづらい\\ \hline
        
        

    \end{tabular}
    \label{free4}
\end{table}

\begin{table}[htbp]
    \centering    \caption{生成したデザインがマジックに適しているかどうかの回答結果}
    \begin{tabular}{|c|p{7em}|p{26em}|} \hline
        被験者区分 & \multicolumn{1}{|c|}{回答} & \multicolumn{1}{|c|}{自由記述} \\ \hline
          &  適している& ふちあり、天地なしを基本的に選んだため \\\cline{3-3}
         A & &上下反転しても同じ柄であり、エッジの部分にもきちんと一定幅の空白があることによって、1デックの中にこっそり表裏反転させたカードをまぜても気づかれないという利点を備えている。\\\cline{3-3}
          & &シンプル目のデザインだったので\\\hline
        & どちらかというと適している& 線対称で違和感がない。デザインを注視されネタを見破られることがなさそう \\ \cline{2-3}
         & どちらでもない&   マジックに適しているかという基準を知らないため \\\cline{3-3}
       B  & & 分からない \\\cline{3-3}
       & &マジシャンではないので\\\cline{3-3}
        & &経験がないので\\ \hline

    \end{tabular}
    \label{free1}
\end{table}

\begin{table}[htbp]
    \centering    \caption{生成したデザインがカーディストリーに適しているかどうかの回答結果}
    \begin{tabular}{|c|p{7em}|p{26em}|} \hline
        被験者区分 & \multicolumn{1}{|c|}{回答} & \multicolumn{1}{|c|}{自由記述} \\ \hline
         A &どちらかという &シンプル寄りなので!\\\cline{3-3}
         &と適している &上下に折り返しても合同だが、左右に折り返すと合同ではなくなる柄であることによって、ディスプレイを行った際に見栄えが良くなるのではないかと感じた。僕の中で「カーディストリーに向いている」と感じるトランプにはこの性質が当てはまることが多い。\\\cline{3-3}
          & &シンプル目のデザインだったので\\ \hline
          &適している&見せる際に生えるから?\\ \cline{3-3}
          & &白い面積が少ないから表裏の判別がしやすい\\\cline{2-3}
        & どちらかというと適している& 線対称で違和感がない。デザインを注視されネタを見破られることがなさそう \\ \cline{3-3}
        & &見栄えはとてもいいデザインができると思うから\\ \cline{2-3}
       B  & どちらでもない&   マジックに適しているかという基準を知らないため \\\cline{3-3}
         & &カーディストリーに適しているかという基準を知らないため\\\cline{3-3}
        & & 分からない \\\cline{2-3}
       & どちらかというと適していない&どちらかというとマジック向きだと思ったから\\ \hline

    \end{tabular}
    \label{free1}
\end{table}



\begin{table}[htbp]
    \centering
    \caption{本システムに追加してほしい機能}
    \begin{tabular}{|c|p{29em}|} \hline
          被験者区分 & \multicolumn{1}{|c|}{自由記述}  \\ \hline
           & 評価のボタンが無段階調整なので大体の場所に合わせれば数字の定位置に自動で止まる、移動するようになると使いやすい \\\cline{2-2}
           & 初めのデックの画像. \\\cline{2-2}
            & ファンやスプレッドしたところを見せる機能 \\\cline{2-2}
           A & 背景や枠をつける機能 \\\cline{2-2}
            & 最初のデザインを増やしたら可能性が広がる\\\cline{2-2}
            & 一番最初の、既存のトランプ柄の中から自分の好みを選ぶフェーズで、色彩情報も好みの判断基準として含まれてしまう恐れがある為、既存のトランプ柄の色彩情報を統一したほうが、柄ベースでの好みの判断がしやすいと感じた。\\ \hline
          & SVGでの出力機能 \\\cline{2-2}
                & 保存時におけるsvg形式でのファイル出力。もしくは解像度(256px, 512px, 1024px, ...)の選択・変更。素材として使いやすくするため。 \\\cline{2-2}
                & 表面デザインも(編集可能かはさておき)用意しておいて、すぐにトランプカードの素材として運用できるように出力されると嬉しい。 \\\cline{2-2}
               B & 色変更のリアルタイム反映 \\\cline{2-2}
                & よくある配色セット \\\cline{2-2}
                & 図形一つ一つに対して細かく色を設定できるといいと思った \\\cline{2-2}
                & 最初にくっきりした線だけのデザインを選ぶと、それからは濃度100の線のデザインしか生成されないので、ときどき半透明の重なりが生まれるといい \\ \hline
    \end{tabular}
    \label{free5}
\end{table}

\begin{table}[htbp]
    \centering
    \caption{本システムに全体に関する意見や感想}
    \begin{tabular}{|c|p{29em}|} \hline
          被験者区分 & \multicolumn{1}{|c|}{自由記述}  \\ \hline
           & システム自体は使いやすかったです。デザインも時々良いと思える物がありますが最初のデザインの選択によってはかなり無理なデザインになることもありました。 \\\cline{2-2}
           &  初めての体験で面白かった\\\cline{2-2}
            &  天地ありのデックも初期選択のデックに追加して欲しいです。\\\cline{2-2}
            & 実際にあるデザインから選ぶより漠然とした線のデザインを組み合わせた方がより、不思議な発想が生み出せる気がします。\\\cline{2-2}
            &とても良いと思います。自分でもデザインについて考えるいい機会になりました。\\\cline{2-2}
            A& 複雑なデザインが見にくく感じました。\\\cline{2-2}
            &既存のトランプ柄などから好みを選択すると、自分好みの柄の候補がたくさんでてくるという体験を初めてしたので、とても面白かったです。\\\cline{2-2}
            &とても良いサービスだと感じました。\\\cline{2-2}
            &参考にはなるが複雑なデザインには向いていない\\\hline
          &  webアプリとして公開される日が今からまちどおしいです。\\\cline{2-2}
                & 初期設定のトランプデザインを自分で追加できるようになると良いと思った。 \\\cline{2-2}
                & 色の操作がむずかしいと思った \\\cline{2-2}
                &全く同じデザインが多数表示される\\\cline{2-2}
                & webアプリとして公開される日が今からまちどおしいです。 \\\cline{2-2}
                & トランプをデザインしようと考えた事がなかったので,普段あまりトランプに関心がない人でも作ってみようと思うきっかけにはいいのではないかと感じた. \\\cline{2-2}
                & 図形一つ一つに対して細かく色を設定できるといいと思った \\ \cline{2-2}
            B    & 4つ以上選択しないといけないのは場合によってつらいシーンがあった。\\\cline{2-2}
                & どうしても掛け合わせたいデザインが2つあって、何回か保存して次のデザインを生成したがかけ合わさってくれなかった。\\ \cline{2-2}
                &UIがんばってください\\\cline{2-2}
                &デザインにそこまでこだわりがないため、よっぽど嫌いなデザインでない限り良いと思ってしまう\\\cline{2-2}
                &染色体で多様なデザインを表現する手法にとても感心しました\\\cline{2-2}
                &デザインをしたことがなかったが,好みのデザインのトランプを簡単に作ることができた.\\\cline{2-2}
                &デザインの変化が分かりやすかったり,様々な模様が生成できていたので対話型遺伝的アルゴリズムと相性のいいテーマだと思った\\ \hline
    \end{tabular}
    \label{free6}
\end{table}

\begin{table}[t]
    \centering
    \caption{ヒアリングの回答}
    \begin{tabular}{|l|}
    \hline
    \multicolumn{1}{|c|}{回答}                              \\ \hline
    生成したデザインをファンやスプレッドで見れる機能が欲しい                                 \\ \hline
    複雑なデザインが多く,組み合わせるとより複雑になってしまい,見にくい                    \\ \hline
    どこかで見たことあるデザインが散見された                                  \\ \hline
    このままのデザインでトランプを作成する人はいないと思った                          \\ \hline
    自分が作りたいデザインのコンセプトが決まっている場合,\\システムのデザインを参考に様々なパターンが考えられる \\ \hline
    シンプルなデザインが物足りなく感じた                                    \\ \hline
    \end{tabular}
    \label{free7}
\end{table}



% \begin{table}[]
    %     \centering
    %     	\caption{生成したデザインに対する購入意欲}
    %     \begin{tabular}{|l|r|}
    %     \hline
    %     \multicolumn{1}{|c|}{回答} & \multicolumn{1}{c|}{割合} \\ \hline
    %     購入したい                    & 5.9%                    \\ \hline
    %     どちらかというと購入したい            & 64.7%                   \\ \hline
    %     どちらかというと購入したくない          & 23.5%                   \\ \hline
    %     購入したくない                  & 5.9%                    \\ \hline
    %     \end{tabular} 
    % \end{table}

    % \begin{table}[]
%     \centering
%     \caption{デザインに対する得意不得意の割合}
%     \begin{tabular}{|l|r|}
%     \hline
%     \multicolumn{1}{|c|}{回答} & \multicolumn{1}{c|}{割合} \\ \hline
%     得意                       & 0%                      \\ \hline
%     どちらかというと得意               & 17.6%                   \\ \hline
%     どちらでもない                  & 17.6%                   \\ \hline
%     どちらかというと不得意              & 41.2%                   \\ \hline
%     不得意                      & 23.5%                   \\ \hline
%     \end{tabular}
% \end{table}