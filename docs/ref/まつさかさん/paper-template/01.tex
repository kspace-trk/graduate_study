\chapter{はじめに}
主要な日常生活動作のひとつとして歩行は重要である.高齢者や片麻痺患者にとってスムーズな歩行はQOLの向上へと繋がるので,歩行の訓練は大切である.また,訓練のモチベーション維持や介護者の負担軽減など,効果的な訓練を実施するために,正確な歩数を簡便に取得できることが望ましい.

一般に歩数の測定には歩数計が用いられている.近年では,スマートフォンに加速度センサが内蔵されており,従来の歩数計の機能がアプリケーションとして実現されている.一般的に,従来の歩数計では,加速度センサを通して3軸加速度信号を取得し,地磁気センサにより検知したデバイスの傾きを補正して歩数計測に利用する.歩行と環境ノイズを区別するための閾値を設定し,設定した閾値を超えた加速度信号の回数を歩数とする.しかし,高齢者や片麻痺患者,杖歩行者など歩行速度が遅い場合や歩行リズムが不規則である場合には,歩行時の加速度が環境ノイズと区別するための閾値を超えない,もしくは歩き始めの検出が不能なため正しい歩数を取得できない.

本研究では主に前者の問題に着目し,従来の歩数計では正確な測定が不能であったケースも含めて,多くの人が精度良く測定できるようにすることを目的とし,新たな歩数計測法を提案する.


% 図はこんな風に入れます.

% \begin{figure}[tbhp]
% \begin{center}
% \includegraphics[scale=0.95]{image/se.eps}
% \caption{共生進化における2つの集団}
% \label{fig:02se}
% \end{center}
% \end{figure}

% 図番号は図\ref{fig:02se}のように参照します.
