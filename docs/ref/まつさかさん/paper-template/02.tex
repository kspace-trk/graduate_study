\chapter{健康と歩数計}
本章では,本研究における背景を説明する.

\section{健康と歩行の関係}
高齢者の健康について,世界保健機構は個々の疾病の有無や併存
疾患の範囲などではなく、生活機能の自立度の度合いといった健康状態に対する包括的評価で判断すべきであると提唱\cite{Who15}している.さらに,2000年に日本で施行された,当時の厚生省が作成した21世紀の日本国民の健康づくり運動方針である健康日本21においても,「健康寿命」という考え方が導入され,健康で自立した日常生活を送る重要性が示唆\cite{Kosei04}されている.
高齢者の廃用性の機能低下を防ぎ生活の質(Quality Of Life: QOL)を高めるため,筋機能のトレーニングは重要である.高齢者に対するトレーニングでは,日常生活動作に関わる運動が積極的に行われている.

\section{従来の歩数計の現状}
歩数計は比較的使用頻度も高く一般普及している健康支援機器であり,計測が簡便で数週間にわたって1日の行動を評価できることから,地域住民を対象とした疫学研究や臨床分野の治療用モニタリング機器として利用されている.また,身体活動量を評価する機器としての有用性も確認されており,高齢者の場合身体活動量の80\%は歩行であり,身体活動量は1日の総歩数と相関関係があるとされている.

初期の歩数計には機械式の歩数カウンタが内蔵されていたが,MEMS技術の向上に伴って,歩数計に加速度センサが使用されるようになった.近年では,ICTの急速な進歩により健康支援機器に大きな変革がもたらされようとしている.Apple社の開発したスマートフォンiPhoneには加速度センサが内蔵されており,スマートフォンを常に携帯することから,アプリケーションとして従来の歩数計の機能が実現されている.
iPhoneには各種センサを束ねる常時稼働の補助プロセッサ,モーションプロセッサが搭載されており,歩数を計測・記録し続けている.現在iPhoneのアプリケーションとして公開されている歩数計の多くは,Apple社が提供するフレームワークであるCore Motion内のCMPedometerというAPIを利用して,モーションプロセッサで記録したデータを呼び出している.Apple社が提供する歩数を取得できるフレームワークとしてHealthKitもあるが,本研究においては歩数の取得だけでなく加速度の取得も同時に行うことが可能なCMPedometerを従来の歩数計として利用する.

加速度センサを用いた一般的な歩数計では,まず3軸加速度信号をデバイスがどのように傾いているかを検知する地磁気センサによって補正し,ノイズ削減などのデータの補正を施す.次に歩行と環境ノイズを区別するための閾値を設定し,設定した閾値を超えた上下の振動を1歩とカウントする.しかし,上記のような歩数計測法では杖歩行や下肢機能障害などにより歩行リズムが不規則である場合や,歩行リズムが不規則である場合に正しい値を得られないことがしばしば見受けられる.特に高齢者において上記の条件が当てはまることが多く,歩行訓練などのリハビリテーション現場や日常生活での活動量評価の際の正確な歩数計測が困難な状況となっている\cite{Sekine07}.
