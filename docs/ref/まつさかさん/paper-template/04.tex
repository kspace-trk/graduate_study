\chapter{粒子群最適化による歩数計測}
粒子群最適化を用いた歩数計測法では,「歩行は周期的な信号であるので,個人に合わせた閾値をはじめに設定することで正確に歩数を測定できるようになる」という仮説に基づき歩数を求める.本計測法は2つのフェーズからなり,まず閾値算出フェーズで粒子群最適化により環境ノイズと歩行の信号を識別するための閾値を求め,次に歩数計測フェーズで各3軸加速度信号が閾値を上回った回数を歩数とする.

\section{粒子群最適化}
粒子群最適化(Particle Swarm Optimization: PSO)は,鳥や魚が群れで効率よく餌を探す行動のうち,以下の点に着目して工学的に模倣した最適解探索アルゴリズムである.
\begin{itemize}
  \item すべての個体が一斉に移動する.
  \item 他の個体の行動に応じて速度を調整しながら移動する.
  \item 餌の在処に関する情報は群れ全体に伝達される.
\end{itemize}

ここで,群れをなして行動する個体を粒子(Particle),粒子の群れを粒子群(Particle swarm)と呼ぶ.粒子群最適化では,問題に対する解を粒子の位置として表現し,複数の粒子を繰り返し移動させてより餌に近い位置を見つけることで最適解を導く.
また,餌への近さを対象問題の解としての良さと考えて各粒子の位置を評価し,より評価の高い位置にいる粒子を良い粒子とする.郡全体として現在時刻までに見つけた最良の位置をグローバルベスト,近傍の最良の個体の位置をローカルベスト,個体自身が現在時刻までに見つけた最良の位置をパーソナルベストとし,以上3つの情報をもとに次の時刻の速度と位置を決定する.

問題の解が$N$個の粒子からなるとき,粒子はN次元空間を移動するものとして,粒子の位置を$N$次元ベクトルで表す.粒子\begin{math}
  P_{i}の位置\vec{x_{i}}のj番目の成分x^{i}_{j}の下限値xmin_{j},および上限xmax_{j}に加え,速度\vec{v_{i}}のj番目の成分v^{i}_jの下限値vmin_{j},および上限値vmax_{j}
 \end{math}を定め,粒子の位置と速度の各成分は式(\ref{xReset}),(\ref{vReset})により初期化する.
 \begin{equation}
   \label{xReset}
   x^{i}_{j} = xmin_{j} + ( xmax_{j} - xmin_{j} ) \times rand[0,1]
 \end{equation}
 \begin{equation}
   \label{vReset}
   v^{i}_{j} = vmin_{j} + ( vmax_{j} - vmin_{j} ) \times rand[0,1]
 \end{equation}
 ただし,\begin{math}
   rand[0,1]
 \end{math}は0以上1以下の実数乱数を生成する関数とする.

基本アルゴリズムのフローチャートを図\ref{fig:04se}に示す.最初に粒子群に含まれる粒子をランダムに生成し,問題の解としての適切さに基づいて評価する.終了条件として,一定回数の繰り返しを終える,目標とする評価値の粒子が得られる,といった条件が用いられる.

\begin{figure}[tbp]
  \begin{center}
  \includegraphics[scale=0.8]{image/PSOalgorithm.png}
  \caption{粒子群最適化の基本アルゴリズム}
  \label{fig:04se}
  \end{center}
\end{figure}

粒子の移動について,まず時刻t+1のときの速さ\begin{math}
  \vec{v_{i}}(t+1)
\end{math}を式(\ref{vt1})により求める.
\begin{equation}
  \label{vt1}
  \vec{v_{i}}(t+1) = I\vec{v_{i}}(t) + Ag\{ \vec{g_{i} }(t) - \vec{x_{i}}(t) \} \times rand[0,1] + Ap\{ \vec{p_{i} }(t) - \vec{ x_{i} }(t)\} \times rand[0,1]
\end{equation}
次に,求めた移動速度に基づいて時刻t+1における位置\begin{math}
  \vec{x_{i}}(t+1)
\end{math}を式(\ref{xt1})により求める.
\begin{equation}
  \label{xt1}
  \vec{x_{i}}(t+1) = \vec{x_{i}}(t) + \vec{v_{i}}(t+1)
\end{equation}
\begin{math}
  Iは慣性係数,A_{g}とA_{p}はそれぞれグローバルベストとパーソナルベストに対する加速係数である.
\end{math}

粒子群最適化の特徴としては以下の3点が挙げられる.
\begin{itemize}
  \item 柔軟な並列処理
  \item 勾配情報を用いない探索
  \item 非線形システムとの対応
\end{itemize}
粒子群最適化における粒子はそれぞれ周囲と情報共有を行うため,効率的に最適解の探索が可能であるといえる.また,勾配情報を用いない探索であるため,不連続な目的関数な場合においても探索が可能である.さらに,粒子群最適化の数理モデルは大規模非線形力学系の一種とみなせるため,多様な最適解探索に利用できる.

\section{閾値算出フェーズ}
閾値算出フェーズにおいては,粒子群最適化により環境ノイズと歩行の信号を識別するための閾値を求める.
本研究における粒子の位置は,3軸それぞれについての閾値である.
また,目測で測定した歩数と候補の閾値から導いた歩数の差を,算出した粒子の位置の評価とした.定数の定義を表\ref{psoSetting}に表す.
\begin{table}[tbp]
  \begin{center}
    \caption{定数の設定}
    \label{psoSetting}
    \begin{tabular}{|l|S[table-format=1.1]|} \hline
      \multicolumn{1}{|c|}{定数名} & {設定値} \\ \hline \hline
      繰り返し数 & {1000} \\ \hline
      粒子群のサイズ & {10} \\ \hline
      慣性係数 & 0.9 \\ \hline
      加速係数(パーソナルベスト) & 0.8 \\ \hline
      加速係数(グローバルベスト) & 0.8 \\ \hline
      標準偏回帰係数の最小値 & -1.0 \\ \hline
      標準偏回帰係数の最大値 & 1.0 \\ \hline
    \end{tabular}
  \end{center}
\end{table}

\section{歩数計測フェーズ}
歩数計測フェーズにおいては,閾値算出フェーズにおいて算出した閾値を用いて歩数を計測する.まず,各時刻において閾値と各軸の加速度を比較し,加速度が閾値を超えたかどうかによりブール値で正規化する.正規化した加速度信号の中で,連続して正である回数を数え,歩数とする.
