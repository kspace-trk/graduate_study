\chapter{動的閾値による歩数計測}
動的閾値を用いた歩数計測法では,Neilの歩数計測アルゴリズム\cite{Neil10}を基に,3軸それぞれの加速度信号に対し,窓長90サンプルごとの最大値と最小値の平均を動的閾値として継続的に更新する.各サンプルにおいて,信号が各窓で設定した動的閾値を上回るか否かを記録し,最後に信号が動的閾値を連続して上回る回数を歩数とする.

\section{動的閾値}
3軸の加速度信号の最大値と最小値を一定サンプルごとに継続的に更新し,平均値を動的閾値レベルとする.動的閾値レベルを更新する頻度は,およそ1周期分に相当する事が望ましいが,1周期の窓長は個人差があるため,本研究においてはサンプルデータとして収集した加速度信号から歩数が最も整合する窓長を動的閾値レベルを更新する頻度とする.
サンプルデータにおいて,動的閾値レベルを更新する頻度と算出された歩数を表\ref{DySetting}に示す.結果から,動的閾値レベルを更新する頻度は90サンプルとする.
\begin{table}[htb]
  \begin{center}
    \caption{動的閾値の範囲}
    \label{DySetting}
    \begin{tabular}{|r|r|} \hline
      \multicolumn{1}{|c|}{窓長} & \multicolumn{1}{|c|}{歩数} \\ \hline \hline
      10 & 53 \\ \hline
      20 & 21 \\ \hline
      30 & 13 \\ \hline
      40 & 15 \\ \hline
      50 & 12 \\ \hline
      60 & 11 \\ \hline
      70 & 14 \\ \hline
      80 & 11 \\ \hline
      90 & 10 \\ \hline
      100 & 12 \\ \hline
      110 & 12 \\ \hline
      120 & 13 \\ \hline
      130 & 10 \\ \hline
      140 & 10 \\ \hline
      150 & 10 \\ \hline
      160 & 12 \\ \hline
    \end{tabular}
  \end{center}
\end{table}

\section{歩数計測法}
90サンプルごとに最大値と最小値を更新する.各時刻において算出した動的閾値レベルと各軸の加速度を比較し,加速度が閾値を超えたかどうかによりブール値で正規化する.正規化した加速度信号の中で,連続して正である回数を数え,歩数とする.動的閾値による歩数計測法のフローチャートを図\ref{fig:Dy}に示す.

\begin{figure}[tbhp]
  \begin{center}
  \includegraphics[scale=0.7]{image/Fig-Dy.png}
  \caption{動的閾値による歩数計測法のフローチャート}
  \label{fig:Dy}
  \end{center}
\end{figure}