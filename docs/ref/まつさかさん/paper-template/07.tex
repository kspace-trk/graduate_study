\chapter{評価実験}
本章では,システムの有用性を示すために行った評価実験の方法と結果について説明する.

\section{実験方法}
従来の歩数計で正確に歩数を測定できない片麻痺患者を被験者として実験する.被験者から事前に収集した一定歩数のiPhone加速度センサーのデータに対し,考案する歩数計測法を用いて歩数を数える.データを収集する際には,従来の歩数計を用いた歩数と実験者が目測で測定した歩数も同時に取得する.
考案した歩数計測法を用いて得られた値(PM)のうち,最も目測の歩数に近い値を基に,目測で得られた歩数(VM)に対する誤差率を算出する.同様に,目測で得られた歩数に対する従来の歩数計による値(CM)の誤差率も算出し,両者の誤差率を比較して本歩数計測法の有用性を示す.誤差率の算出方法は式(\ref{gosaritu})の通りである.ただし,粒子群最適化を用いた歩数計測法については,あるデータを基に算出した閾値を用いて他データの歩数を測定することとする.

PMとCMのそれぞれの誤差の比較には各誤差率の平均,標準偏差に加えて,対応のあるt検定を行った.有意水準は5\%とした.
\begin{equation}
  \label{gosaritu}
  Error(\%) = 100 \times (PM or CM - VM) / VM
\end{equation}

\section{実験結果}
実験結果を表\ref{result},\ref{resultT}に表す.また,使用したデータと各歩数計測法による歩数の計測結果を付録Aに掲載する.

被験者のすべての歩行について,従来の歩数計では測定不能であった.考案した手法の中では動的閾値を用いた歩数計測法が標準偏差・誤差率平均において有意に小さい値を示した.

実行時間は自己相関関数を用いた歩数計測法,動的閾値を用いた歩数計測法それぞれについて平均1.15秒,1.69秒であり,粒子群最適化による閾値算出フェーズでは約16時間,歩数計測フェーズでは0.86秒であった.

\begin{table}[htb]
  \begin{center}
    \caption{各手法の実験結果}
    \label{result}
    \begin{tabular}{|l|c|c|c|} \hline
       & 標準偏差 & VMの誤差率平均 & CMの誤差率平均  \\ \hline \hline
      粒子群最適化による歩数計測法 & 0.35 & 71\% &  \\ \cline{1-3}
      自己相関による歩数計測法 & 0.11 & 29\% & 100\% \\ \cline{1-3}
      動的閾値による歩数計測法 & 0.10 & 14\% &  \\ \hline
    \end{tabular}
  \end{center}
\end{table}

\begin{table}[htb]
  \begin{center}
    \caption{各手法のt検定による比較}
    \label{resultT}
    \begin{tabular}{|l|S[table-format=1.1e+1]|} \hline
       & {p値}  \\ \hline \hline
      粒子群最適化による歩数計測法 & 3.5e-1  \\ \hline
      自己相関による歩数計測法 & 8.1e-6 \\ \hline
      動的閾値による歩数計測法 & 5.2e-1  \\ \hline
    \end{tabular}
  \end{center}
\end{table}

