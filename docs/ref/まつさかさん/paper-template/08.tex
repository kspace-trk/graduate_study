\chapter{考察}
「歩行は周期的な信号であるので,個人に合わせた閾値をはじめに設定することで正確に歩数を測定できるようになる」という仮説を立てたが,被験者の体調によって歩幅や歩行速度が異なるために歩行を一定の閾値によって測定することは困難であると考えられる.粒子群最適化による歩数計測法については前述の理由に加えて,閾値算出の実行時間が平均約16時間かかるため,実用性に乏しいと考えられる.

自己相関関数による歩数計測法についても,歩数計としては高精度とならなかった.自己相関関数は感度が高く,ポケットにiPhoneを入れる際の振動など,ノイズの波形が歩行中の信号と偶然類似している場合にも歩数と数えていることが原因であると考えられる.

動的閾値による歩数計測法は本研究において最も良い結果を記録した.従来の歩数計においても使用されることのある計測法であるが,動的閾値の算出窓長を変更することで精度が変わる.本研究では評価実験に用いたデータとは別のデータにおいて窓長90サンプルが最も良い結果が出たため90サンプルを窓長として使用しているが,個々人に対応するために例えば「自己相関関数を算出する過程で得られる信号の周期を動的閾値の窓長とする」など工夫する必要があると考えられる.

今後の課題として,歩き始めの検出についての検討を重ねる必要があることが挙げられる.しかしながら,被験者のデータにおいて,従来の歩数計では計測が不能であったが,本研究で提案した3つの歩数計測法すべてで歩行を確認することができた.本手法を用いることで,従来の歩数計で測定不能であった対象者の正確な歩数計測が可能となり,より効果的な歩行訓練の提供,モチベーションの向上,および健康増進へ繋がると期待される.
