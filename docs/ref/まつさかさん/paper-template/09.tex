\chapter{おわりに}
本研究は,従来の歩数計では正確な測定が不能であった高齢者や片麻痺患者を対象とし,歩数計の測定精度を向上させるための新たな歩数計測法を提案することを目的とした.提案する手法は,3軸加速度信号から粒子群最適化を用いて閾値を算出し,得た閾値を基に歩数を数える方法,自己相関関数を用いて周期性の有無を確認し歩数を数える方法,動的閾値を用いて歩数を数える方法の3つがある.
杖歩行の片麻痺患者において測定精度を検討したところ,従来の歩数計ではすべての歩行について測定不能であったことに対し,本研究で提案する手法では粒子群最適化を用いた歩数計測法,自己相関関数を用いた歩数計測法,動的閾値を用いた歩数計測法それぞれについて誤差は71\%,29\%,14\%であり,歩行を確認することができた.本研究で提案する手法は,杖歩行時のみならず,歩行速度が遅い,または下肢機能障害を持つなど従来の歩数計で困難であった高齢者の歩数検出率を高めることが可能であり,より効果的な歩行訓練の提供,モチベーションの向上,および健康増進へ繋がると期待される.