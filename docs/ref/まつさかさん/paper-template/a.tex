\chapter{実験結果一覧}
評価実験で使用したデータおよび実験結果を表\ref{data}から表\ref{data5}に示す.
なお,実験結果の番号は使用したデータの番号である.

\begin{table}[h]
  \caption{使用データ}
  \renewcommand{\arraystretch}{1.1}
  \begin{tabular}{|c|c|c|c|}
    \hline \rm & データ & \rm CM & VM \\ \hline
    \hline
    1 & \begin{minipage}{0.8\hsize}
      \centering
      \includegraphics[width=120mm]{image/01.png}
    \end{minipage} &  0  &  20 \\ \hline
    2 & \begin{minipage}{130mm}
      \centering
      \scalebox{0.5}{\includegraphics{image/02.png}}
    \end{minipage} & 0 & 20  \\ \hline
    3 & \begin{minipage}{130mm}
      \centering
      \scalebox{0.5}{\includegraphics{image/03.png}}
    \end{minipage} & 0 &  20  \\ \hline
  \end{tabular}
  \label{data}
\end{table}

\begin{table}[h]
  \renewcommand{\arraystretch}{1.1}
  \begin{tabular}{|c|c|c|c|}
    \hline \rm & データ & \rm CM & VM \\ \hline
    \hline
    4 & \begin{minipage}{130mm}
      \centering
      \scalebox{0.5}{\includegraphics{image/04.png}}
    \end{minipage} & 0 & 20  \\ \hline
    5 & \begin{minipage}{0.8\hsize}
      \centering
      \includegraphics[width=120mm]{image/05.png}
    \end{minipage} &  0  &  20 \\ \hline
    6 & \begin{minipage}{130mm}
      \centering
      \scalebox{0.5}{\includegraphics{image/06.png}}
    \end{minipage} & 0 & 20  \\ \hline
    7 & \begin{minipage}{130mm}
      \centering
      \scalebox{0.5}{\includegraphics{image/07.png}}
    \end{minipage} & 0 &  20  \\ \hline
  \end{tabular}
\end{table}

\begin{table}[h]
  \renewcommand{\arraystretch}{1.1}
  \begin{tabular}{|c|c|c|c|}
    \hline \rm & データ & \rm CM & VM \\ \hline
    \hline
    8 & \begin{minipage}{130mm}
      \centering
      \scalebox{0.5}{\includegraphics{image/08.png}}
    \end{minipage} & 0 & 20  \\ \hline
    9 & \begin{minipage}{0.8\hsize}
      \centering
      \includegraphics[width=120mm]{image/09.png}
    \end{minipage} &  0  &  20 \\ \hline
    10 & \begin{minipage}{130mm}
      \centering
      \scalebox{0.5}{\includegraphics{image/10.png}}
    \end{minipage} & 0 & 20  \\ \hline
    11 & \begin{minipage}{130mm}
      \centering
      \scalebox{0.5}{\includegraphics{image/11.png}}
    \end{minipage} & 0 &  20  \\ \hline
  \end{tabular}
\end{table}

\begin{table}[h]
  \renewcommand{\arraystretch}{1.1}
  \begin{tabular}{|c|c|c|c|}
    \hline \rm & データ & \rm CM & VM \\ \hline
    \hline
    12 & \begin{minipage}{130mm}
      \centering
      \scalebox{0.5}{\includegraphics{image/12.png}}
    \end{minipage} & 0 & 20  \\ \hline
    13 & \begin{minipage}{0.8\hsize}
      \centering
      \includegraphics[width=120mm]{image/13.png}
    \end{minipage} &  0  &  20 \\ \hline
    14 & \begin{minipage}{130mm}
      \centering
      \scalebox{0.5}{\includegraphics{image/14.png}}
    \end{minipage} & 0 & 20  \\ \hline
    15 & \begin{minipage}{130mm}
      \centering
      \scalebox{0.5}{\includegraphics{image/15.png}}
    \end{minipage} & 0 &  20  \\ \hline
  \end{tabular}
\end{table}

\begin{table}[h]
  \renewcommand{\arraystretch}{1.1}
  \begin{tabular}{|c|c|c|c|}
    \hline \rm & データ & \rm CM & VM \\ \hline
    \hline
    16 & \begin{minipage}{130mm}
      \centering
      \scalebox{0.5}{\includegraphics{image/16.png}}
    \end{minipage} & 0 & 20  \\ \hline
    17 & \begin{minipage}{0.8\hsize}
      \centering
      \includegraphics[width=120mm]{image/17.png}
    \end{minipage} &  0  &  20 \\ \hline
    18 & \begin{minipage}{130mm}
      \centering
      \scalebox{0.5}{\includegraphics{image/18.png}}
    \end{minipage} & 0 & 20  \\ \hline
    19 & \begin{minipage}{130mm}
      \centering
      \scalebox{0.5}{\includegraphics{image/19.png}}
    \end{minipage} & 0 &  20  \\ \hline
  \end{tabular}
\end{table}

\begin{table}[h]
  \renewcommand{\arraystretch}{1.1}
  \begin{tabular}{|c|c|c|c|}
    \hline \rm & データ & \rm CM & VM \\ \hline
    \hline
  20 & \begin{minipage}{130mm}
  \centering
  \scalebox{0.5}{\includegraphics{image/20.png}}
  \end{minipage} & 0 & 20  \\ \hline
\end{tabular}
\end{table}

\begin{table}[htb]
  \begin{center}
    \caption{粒子群最適化の閾値}
    \label{data1}
    \begin{tabular}{|c|c|c|c|} \hline
       & x軸 & y軸 & z軸  \\ \hline \hline
      1回目 & 0.092 & -0.894 & -0.0523  \\ \hline
      2回目 & 0.079 & -0.820 & -0.0148 \\ \hline
    \end{tabular}
  \end{center}
\end{table}

\begin{table}[htb]
  \begin{center}
    \caption{1回目の粒子群最適化による閾値を用いた歩数計測法の結果}
    \label{data2}
    \begin{tabular}{|c|c|c|c|} \hline
      & x軸 & y軸 & z軸  \\ \hline \hline
      11 & 0 & 43 & 32  \\ \hline
      12 & 0 & 61 & 41 \\ \hline
      13 & 0 & 62 & 55 \\ \hline
      14 & 0 & 64 & 44 \\ \hline
      15 & 0 & 81 & 44 \\ \hline
      16 & 0 & 59 & 45 \\ \hline
      17 & 0 & 55 & 33 \\ \hline
      18 & 0 & 56 & 23 \\ \hline
      19 & 0 & 61 & 33 \\ \hline
      20 & 0 & 58 & 22 \\ \hline
    \end{tabular}
  \end{center}
\end{table}

\begin{table}[htb]
  \begin{center}
    \caption{2回目の粒子群最適化による閾値を用いた歩数計測法の結果}
    \label{data3}
    \begin{tabular}{|c|c|c|c|} \hline
      & x軸 & y軸 & z軸  \\ \hline \hline
      11 & 0 &	 2 &	33  \\ \hline
      12 & 0 &	57 &	51  \\ \hline
      13 & 0 &	58 &	52  \\ \hline
      14 & 0 &	55 &	54  \\ \hline
      15 & 0 &	26 &	23  \\ \hline
      16 & 0 &	69 &	58  \\ \hline
      17 & 0 &	55 &	48  \\ \hline
      18 & 0 &	25 &	21  \\ \hline
      19 & 0 &	36 &	45  \\ \hline
      20 & 0 &	11 &	30    \\ \hline
    \end{tabular}
  \end{center}
\end{table}

\begin{table}[htb]
  \begin{center}
    \caption{自己相関関数を用いた歩数計測法の結果}
    \label{data4}
    \begin{tabular}{|c|c|c|c|} \hline
      &x軸 & y軸 & z軸  \\ \hline \hline
      1 &	15 &	15 &	15  \\ \hline
      2 &	18 &	19 &	19  \\ \hline
      3 &	22 &	23 &	23  \\ \hline
      4 &	24 &	25 &	25  \\ \hline
      5 &	22 &	23 &	24  \\ \hline
      6 &	15 &	14 &	15  \\ \hline
      7 &	13 &	12 &	13  \\ \hline
      8 &	13 &	14 &	14  \\ \hline
      9 &	13 &	14 &	14  \\ \hline
      10 &	14 &	13 &	14  \\ \hline
      11 &	12 &	13 &	13  \\ \hline
      12 &	13 &	12 &	13  \\ \hline
      13 &	14 &	13 &	14  \\ \hline
      14 &	14 &	14 &	14  \\ \hline
      15 &	13 &	13 &	12  \\ \hline
      16 &	12 &	13 &	12  \\ \hline
      17 &	12 &	12 &	12  \\ \hline
      18 &	10 &	10 &	11  \\ \hline
      19 &	11 &	11 &	13  \\ \hline
      20 &	11 &	11 &	11  \\ \hline
    \end{tabular}
  \end{center}
\end{table}

\begin{table}[htb]
  \begin{center}
    \caption{動的閾値を用いた歩数計測法の結果}
    \label{data5}
    \begin{tabular}{|c|c|c|c|} \hline
      &x軸 & y軸 & z軸  \\ \hline \hline
      1 &	15 &	23 &	16  \\ \hline
      2 &	24 &	27 &	22  \\ \hline
      3 &	26 &	31 &	25  \\ \hline
      4 &	30 &	46 &	27  \\ \hline
      5 &	25 &	35 &	22  \\ \hline
      6 &	20 &	21 &	16  \\ \hline
      7 &	18 &	20 &	12  \\ \hline
      8 &	18 &	20 &	17  \\ \hline
      9 &	17 &	18 &	15  \\ \hline
      10 &	23 &	21 &	15  \\ \hline
      11 &	17 &	22 &	13  \\ \hline
      12 &	15 &	16 &	13  \\ \hline
      13 &	16 &	20 &	18  \\ \hline
      14 &	16 &	17 &	15  \\ \hline
      15 &	16 &	14 &	15  \\ \hline
      16 &	13 &	19 &	14  \\ \hline
      17 &	13 &	16 &	12  \\ \hline
      18 &	11 &	14 &	11  \\ \hline
      19 &	16 &	16 &	10  \\ \hline
      20 &	10 &	16 &	13  \\ \hline
    \end{tabular}
  \end{center}
\end{table}