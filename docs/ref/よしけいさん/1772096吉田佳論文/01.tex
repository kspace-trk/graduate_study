\chapter{はじめに}
現在,個人の感性を反映した楽曲の自動生成に関する研究が進められている.
大谷らが開発したシステム\cite{otani16}では,ユーザに指定された
既存楽曲から個人の感性モデルを獲得し,進化計算アルゴリズムにより
感性に即したメロディを生成して楽曲を出力する.
出力される楽曲はメロディと和音進行で構成されている.
現状では,和音進行をもとに伴奏部を作成する際に同時和音のみを使用している.
同時和音とは和音の構成音を同時に演奏することである.
同時和音を多用する楽曲ではリズム感が欠如する.
一般的な楽曲の伴奏部では,同時和音のみならず分散和音も用いられる.
分散和音とは和音の構成音を分けて演奏することである.
拍子感が増えることにより,楽曲にリズム感を与えることができる.
伴奏部に同時和音のみを使用している楽曲では,個人の感性が反映されていても,リズム感に欠け,印象が悪くなると考えられる.

本研究では,適切なリズム感を持った楽曲の生成を目的とし,メロディに合う分散和音を取り入れた楽曲の生成手法を提案する.
