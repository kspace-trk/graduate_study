\chapter{考察}
楽曲Dでは,リズム感の良さと伴奏の合致度について提案手法より(c),(d)の方が評価値の平均が高い.
自由記述では「フレーズの切れ目にそった伴奏に良い評価をする傾向があった」という意見が挙げられた.
楽曲Dには,1小節あたりの和音の移り変わりが多い特徴がある.現状ではルール適用の際,フレーズの切れ目を考慮していない.フレーズの切れ目では,音高の変化が大きい方がメロディに合うと考えられる.

%「伴奏とメロディの音域が離れすぎているものは悪く評価する影響があった」という意見があげられた.

提案手法において,楽曲Eではリズム感の良さについて適用前の音源より高い評価が得られていることから,リズム感をもった楽曲を生成できているといえる.
しかし,伴奏の適合度については2.1と他の音源に比べ大幅に低い.自由記述では「分散の伴奏の時にメロディと音がぶつかる」との指摘があった.
提案手法では,音の衝突を回避することを考慮していない.メロディと伴奏部の音の衝突は評価を下げると考えられる.

楽曲Lでは,提案手法よりも他の分散和音を適用した音源の方がリズム感の良さの評価が高い.
自由記述では「伴奏によって雰囲気が変わる」という意見が複数挙げられた.
伴奏が楽曲全体の雰囲気を変え,評価に影響を与えたと考えられる.

適用前の音源において,楽曲Jではリズム感の良さについて他の音源より評価が低いので,リズム感に欠けているといえる.
しかし楽曲Jでは,伴奏の適合度について平均が4.2と適用前の音源における評価値が提案手法を上回った.
自由記述では「遅めのメロディに対して細かく刻むような伴奏の楽曲は良い印象を抱かなかった」という意見が挙がった.
曲Mのメロディは1小節あたりの音数が少ない.現状では伴奏部を生成する際,音数を考慮していない.
メロディの音数が少ない部分では,伴奏部の音数が多いと,メロディのリズムを阻害すると考えられる.
