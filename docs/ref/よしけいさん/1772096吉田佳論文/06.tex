\chapter{おわりに}
本研究では,適切なリズム感を持った楽曲の生成手法を提案した.
評価実験では,「伴奏によってかなり雰囲気が変わる」という意見が複数得られ,
メロディの雰囲気を保持したまま編曲する必要があることがわかった.
分散和音の標準的な型や,学習データの数を増やすことで編曲の制度を向上させることができる.
また,「遅めのメロディに対して細かく刻むような伴奏の楽曲は良い印象を抱かなかった」という意見も得られた.
メロディの音数に対応した伴奏部を生成することで,
より適切なリズム感を持った楽曲の生成が可能になると考えられる.