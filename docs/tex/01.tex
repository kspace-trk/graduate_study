\chapter{はじめに}
Progressive Houseは,Electronic Dance Music(以下EDM)のサブジャンルのひとつである\cite{Sonja18}.EDMとは,Digital Audio Workstation(以下DAW)やシンセサイザを用いて作曲し,人々を踊らせることを目的とした楽曲のジャンルである.EDMジャンルの中でもProgressive Houseは,Beats Per Minute(以下BPM)が128前後のテンポで,ベースやパッド,ドラムなどを演奏し,サビ部分で高音の電子音のリードを短いメロディパターンで繰り返し演奏する点が特徴である.メロディパターンは,音高パターンとリズムパターンの組み合わせで表現される.一般的なProgressive Houseのリードにおいて,音高パターンは4小節ごとに繰り返され,リズムパターンは1〜2小節ごとに繰り返される.したがって,Progressive Houseのメロディを考案する場合,短いメロディパターンを考える必要がある.作曲家による一般的な作業手順では,はじめにサビのメロディを考案し,メロディに基づいたスケールからベースやパッドを考案する.メロディ考案時には有名な既存曲を参考にすることが多いため,作曲したメロディが有名な既存曲と類似する可能性がある.短いメロディパターンを繰り返す点が特徴であることから,一部が類似すると曲全体が類似し,独自性に欠けるという問題点が挙げられる.

本研究では,Progressive Houseの作曲におけるメロディの独自性向上,および作業時間の短縮を目的として,ユーザの感性に基づいたメロディ生成システムを構築する.




図はこんな風に入れます.

\begin{figure}[tbhp]
\begin{center}
\includegraphics[scale=0.95]{image/se.eps}
\caption{共生進化における2つの集団}
\label{fig:02se}
\end{center}
\end{figure}

図番号は図\ref{fig:02se}のように参照します.
