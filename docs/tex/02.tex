\chapter{Progressive House}
Progressive Houseは,Electronic Dance Music(以下 EDM)のサブジャンルのひとつである.本章では,本研究における背景を説明する.

\section{Progressive Houseの特徴}
Progressive Houseはブレイクダウン,ビルドアップ,ドロップの3つで構成される.
ブレイクダウンは,曲の中で一番穏やかなバートである.一般的にはボーカルやコード,および雰囲気作りのためのループ素材等で構成される.美しい雰囲気を想像させるために,リバーブやディレイといったサウンドエフェクトが用いられる.
ビルドアップは,ブレイクダウンからドロップにかけての盛り上がりを演出するパートである.ブレイクダウンにドラムが加わることで構成される.
ドロップは,曲の中で一番盛り上がるパートである.一般的に以下7つの楽器で構成される.
\begin{itemize}
  \item ベース
  \item パッド
  \item リード
  \item アルペジオ
  \item ドラム
  \item エフェクト音
\end{itemize}
特に,高音のリードを短いメロディパターンで繰り返し演奏するのが特徴である.

\section{作曲における問題点}
作曲家による一般的な作業手順では,はじめにリードのメロディを考案し,メロディに基づいたスケールからベースやパッド等を考案する.一般的なProgressive Houseのリードにおいて,音高パターンは4小節ごとに繰り返され,リズムパターンは1~2小節ごとに繰り返される.したがって,Progressive House のメロディを考案する場合,短いメロディパターンを考える必要がある.メロディ考案時には有名な既存曲を参考にすることが多いため,作曲したメロディが有名な既存曲と類似する可能性がある.
