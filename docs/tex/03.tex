\chapter{遺伝的アルゴリズム}

\section{概要}
遺伝的アルゴリズム(Genetic Algorithm;GA)は、生物が環境に適応し進化する過程を模倣した最適解探索アルゴリズムである。問題に対する解を個体の染色体とし、解の構成要素を遺伝子として表現する。個体の評価、交叉、突然変異によって複数の個体を進化させ,環境に適応できる個体を見つけることで最適解を導く。基本アルゴリズムのフローチャートを図3.1に示す
%TODO フローチャート書く 本14pに親個体選択を追加

\section{染色体}
生物の各個体の形質や遺伝情報は、遺伝子によって決定される。
GAでは、解に関する情報を示す値を遺伝子と呼び、遺伝子を配列にしたものを染色体と呼ぶ。
遺伝子が配置される位置を表す番号を遺伝子座、配列として表現される遺伝子の構成を遺伝子型、遺伝子から発現した形質を表現型と呼ぶ。

\section{個体の評価}
染色体をもとに構成された表現型は個体と呼ばれ、各個体の解としての良さが評価される。評価結果を表す数値を適応度、適応度を求める関数を適応度関数と呼ぶ。

\section{次世代生成}
GAでは、よりよい個体群を生成するために、個体群内のすべての個体の適応度を算出し、算出された適応度をもとに次世代の個体群を生成する。

\subsection{親個体選択}

\section{交叉と突然変異}
交叉と突然変異による次世代個体生成

