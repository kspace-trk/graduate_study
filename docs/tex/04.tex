\chapter{システム構成}
本研究では、Progressive Houseのサビメロディを生成するシステムを構築した。

\section{メイン画面}
あああああああああああああああああああああああああああああああああああああああああああああああああああああああああああああああああああああああああああああああああああああああああああああああああああああああああああああああああああああああああああああああああああああああああああああああ

\section{メロディ再生画面}
ああああああああ


\section{メロディルールの獲得}
メロディルールとは,メロディ生成においてProgressive Houseらしさを表現する際に使用するメロディの特徴データである.
メロディルールの構成を書き、それぞれ実例をもとに詳しく説明する。
\subsection{音高差分データ}
\subsection{リズムデータ}
\subsection{メロディ変異データ}
\subsection{メロディ繰り返しデータ}
実例の楽譜画像を用いて説明する。

\section{メロディの生成}
メロディ生成手順を示す。
概要を参考に箇条書きで書く


\section{メロディ評価部}
あああああああああああああああああああああああああああああああああああああああああああああああああああああああああああああああああああああああああああああああああああああああああああああああああああああああああああああああああああああああああああああああああああああああああああああああ

