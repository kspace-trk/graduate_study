\chapter{システム構成}
本章では,本研究で構築したProgressive Houseのメロディ生成支援システムについて説明する.

\section{メロディルールの獲得}
メロディルールとは,メロディ生成においてProgressive Houseらしさを表現する際に使用するメロディの特徴データである.音高差分データ,リズムデータ,メロディ変異データ,およびメロディ繰り返し回数データで構成され,有名なProgressive Houseの既存曲のサビ部分冒頭16小節のみを学習データとして獲得される.

音高差分データとは,学習データに含まれる各音とキーの音高の差を表した数値である.
リズムデータとは,学習データに含まれる各音の音価を表した数値である.
変異データとは,学習データのメロディを繰り返し回数分に分割し,それぞれの音数や音高,リズムを比較して算出した差分である.

繰り返し回数データとは,学習データのメロディ内における同じメロディの繰り返し回数である.繰り返し回数は,音高とリズムの類似度から算出する.はじめにメロディを4小節ごとに分割し,1小節目のメロディと,2,3,4小節目のメロディを比較し,一致している割合を類似度として算出する.類似度が60\%以上の場合,繰り返し回数は4回とする.60\%未満の場合は,学習データのメロディの前半8小節と後半8小節の類似度を算出する.60\%以上一致している場合,繰り返し回数は2回とする.すべてに該当しないメロディの繰り返し回数は0回とする.

\section{ルールを適用したメロディの生成}
図4.1のような画面で,繰り返し回数やキーを設定する.メロディルールに基づいた初期メロディを生成手順を以下に示す.
\begin{itemize}
  \item 繰り返し回数に基づいて音数をランダムに決定する.
  \item リズム,協和音となる音高をランダムに決定する.
  \item 繰り返し回数分繰り返し,16小節のメロディを生成する.
  \item 変異データに基づいてメロディを変異させる.以上の手順で生成した初期世代のメロディの冒頭4小節を図4.1に示す.
\end{itemize}

\section{評価および次世代生成}
初期世代以降のメロディを生成するために,ユーザの評価値を適応度とした世代交代を行う.ユーザの好みの音高変化を次世代へ反映するために,遺伝子の並びの多くを次世代に継承できる一点交叉を採用する.図4.1のメロディから生成された次世代メロディの例を図4.2に示す.

\vskip\baselineskip
\vskip\baselineskip
\begin{figure}[htbp]
	\begin{center}
		\includegraphics[scale=0.7]{image/init.png}
		\caption{初期世代メロディ}
	\end{center}
\end{figure}

\begin{figure}[htbp]
	\begin{center}
		\includegraphics[scale=0.7]{image/nextInd.png}
		\caption{次世代メロディ}
	\end{center}
\end{figure}

