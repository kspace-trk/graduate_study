\chapter{システム構成}
本章では、本研究で構築したProgressive Houseのメロディ生成支援システムについて説明する。

\section{メロディルールの獲得}
メロディルールとは,メロディ生成においてProgressive Houseらしさを表現する際に使用するメロディの特徴データである.音高差分データ,リズムデータ,メロディ変異データ,およびメロディ繰り返し回数データで構成され,有名なProgressive Houseの既存曲のサビ部分冒頭16小節のみを学習データとして獲得される.
\subsection{音高差分データ}
音高差分データとは,学習データに含まれる各音とキーの音高の差を表した数値である.
\subsection{リズムデータ}
リズムデータとは,学習データに含まれる各音の音価を表した数値である.
\subsection{メロディ変異データ}
変異データとは,学習データのメロディを繰り返し回数分に分割し,それぞれの音数や音高,リズムを比較して算出した差分である.
\subsection{メロディ繰り返しデータ}
繰り返し回数データとは,学習データのメロディ内における同じメロディの繰り返し回数である.繰り返し回数は,音高とリズムの類似度から算出する.はじめにメロディを4小節ごとに分割し,1小節目のメロディと,2,3,4小節目のメロディを比較し,一致している割合を類似度として算出する.類似度が60\%以上の場合,繰り返し回数は4回とする.60\%未満の場合は,学習データのメロディの前半8小節と後半8小節の類似度を算出する.60\%以上一致している場合、繰り返し回数は2回とする.すべてに該当しないメロディの繰り返し回数は0回とする.

\section{メロディの生成}
メロディ生成手順を示す。
概要を参考に箇条書きで書く


\section{メロディ評価部}
あああああああああああああああああああああああああああああああああああああああああああああああああああああああああああああああああああああああああああああああああああああああああああああああああああああああああああああああああああああああああああああああああああああああああああああああ

