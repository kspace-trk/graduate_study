\chapter{評価実験}
本章では、構築したシステムの有用性を示すために行った評価実験の内容と結果について説明する。

\section{実験内容}
レーベルからリリース経験のあるProgressive House作曲家7名を被験者として,評価実験を実施した.システムを使用させた上で,以下の項目についてメロディの独自性や作曲時間の短縮見込み時間,作曲意欲の変化などの9項目をアンケート形式で回答させた.
\begin{itemize}
  \item アーティスト名義を教えてください
  \item 作曲歴(自由記述)
  \item 1曲を完成させるのに、普段はどれくらいの時間がかかっているか(自由記述)
  \item 本システムによって生成されたメロディーは独自性が高かったか
  \item 本システムを使用することで、メロディーアイデアが思い浮かんだか
  \item 本システムを使用することで、作曲意欲が向上したか
  \item 本システムが作曲活動に役に立つ感じたか
  \item 本システムを使用することによって見込まれる作曲の短縮時間(自由記述)
  \item 自由記述感想
\end{itemize}
本システムの画面を付録Aに、本実験で用いたGoogleフォームの画面を付録Bに示す。

\section{実験結果}
回答した被験者を番号1-7とし、作曲歴を表5.1に、1曲あたりの作曲時間を表5.2に、生成されたメロディの独自性の高さを表5.3に、メロディアイデアが思い浮かんだかを表5.4に、作曲意欲が向上したかを表5.5に、作曲活動に役に立つかを表5.6に、見込まれる作曲の短縮時間を表5.7に、自由記述感想を表5.8に表す。

\begin{table}[htbp]
  \begin{center}
    \caption{作曲歴についての回答}
    \begin{tabular}{cc}
      \hline
      被験者番号 & 回答 \\ \hline \hline
      1 & 2年5か月\rule[-3mm]{0mm}{8mm} \\ \hline
      2 & 5年\rule[-3mm]{0mm}{8mm} \\ \hline
      3 & 2年半\rule[-3mm]{0mm}{8mm} \\ \hline
      4 & 6年\rule[-3mm]{0mm}{8mm} \\ \hline
      5 & 4~5年\rule[-3mm]{0mm}{8mm} \\ \hline
      6 & 4年\rule[-3mm]{0mm}{8mm} \\ \hline
      7 & 3年\rule[-3mm]{0mm}{8mm} \\ \hline
    \end{tabular}
  \end{center}
\end{table}

\begin{table}[htbp]
  \begin{center}
    \caption{1曲あたりの作曲時間についての回答}
    \begin{tabular}{cp{30em}}
      \hline
      被験者番号 & 回答\rule[-3mm]{0mm}{8mm} \\ \hline \hline
      1 & メロディーを作るのにかかる時間は毎回ちがうので言えませんが、メロディーができてからの作業時間は12~24時間くらいです。\rule[-3mm]{0mm}{8mm} \\ \hline
      2 & 72時間位\rule[-3mm]{0mm}{8mm} \\ \hline
      3 & 30〜50時間\rule[-3mm]{0mm}{8mm} \\ \hline
      4 & 10時間\rule[-3mm]{0mm}{8mm} \\ \hline
      5 & 早くて1週間~2週間\rule[-3mm]{0mm}{8mm} \\ \hline
      6 & 早くて一日\rule[-3mm]{0mm}{8mm} \\ \hline
      7 & 1時間\rule[-3mm]{0mm}{8mm} \\ \hline
    \end{tabular}
  \end{center}
\end{table}

\begin{table}[htbp]
  \begin{center}
    \caption{生成されたメロディの独自性の高さについての回答}
    \begin{tabular}{|c|p{10em}|c|}
      \hline
      回答 & 被験者番号 & 割合\rule[-3mm]{0mm}{8mm} \\ \hline \hline
      はい & 1, 4, 6, 7 & 57.1\% \rule[-3mm]{0mm}{8mm} \\ \hline
      いいえ & 2, 3 & 28.6\% \rule[-3mm]{0mm}{8mm} \\ \hline
      その他 & 5 & 14.3\% \rule[-3mm]{0mm}{8mm} \\ \hline
    \end{tabular}
  \end{center}
\end{table}

\begin{table}[htbp]
  \begin{center}
    \caption{メロディアイデアが思い浮かんだかについての回答}
    \begin{tabular}{|c|p{10em}|c|}
      \hline
      回答 & 被験者番号 & 割合\rule[-3mm]{0mm}{8mm} \\ \hline \hline
      はい & 1, 2, 3, 5, 6, 7 & 85.7\% \rule[-3mm]{0mm}{8mm} \\ \hline
      いいえ & 4 & 14.3\% \rule[-3mm]{0mm}{8mm} \\ \hline
    \end{tabular}
  \end{center}
\end{table}

\begin{table}[htbp]
  \begin{center}
    \caption{作曲意欲が向上したかについての回答}
    \begin{tabular}{|c|p{10em}|c|}
      \hline
      回答 & 被験者番号 & 割合\rule[-3mm]{0mm}{8mm} \\ \hline \hline
      はい & 1, 4, 5, 6, 7 & 71.4\% \rule[-3mm]{0mm}{8mm} \\ \hline
      いいえ & 2, 3 & 28.6\% \rule[-3mm]{0mm}{8mm} \\ \hline
    \end{tabular}
  \end{center}
\end{table}

\begin{table}[htbp]
  \begin{center}
    \caption{作曲活動に役に立つかついての回答}
    \begin{tabular}{|c|p{10em}|c|}
      \hline
      回答 & 被験者番号 & 割合\rule[-3mm]{0mm}{8mm} \\ \hline \hline
      はい & 1, 2, 3, 4, 5, 6, 7 & 100\% \rule[-3mm]{0mm}{8mm} \\ \hline
      いいえ &  & 0\% \rule[-3mm]{0mm}{8mm} \\ \hline
    \end{tabular}
  \end{center}
\end{table}

\begin{table}[htbp]
  \begin{center}
    \caption{見込まれる作曲の短縮時間についての回答}
    \begin{tabular}{cp{30em}}
      \hline
      被験者番号 & 回答\rule[-3mm]{0mm}{8mm} \\ \hline \hline
      1 & 僕の場合、半日くらいたっても全然メロディーが思いつかなくて時間を無駄にしてしまうことがあるのですが、このシステムで作られたメロディーからインスピレーションを受ければ10分前後でいいメロディーが作れるのでは?と思いました!\rule[-3mm]{0mm}{8mm} \\ \hline
      2 & 全作曲時間の20%程だが、時間短縮と言うよりは曲の数が増えるイメージ\rule[-3mm]{0mm}{8mm} \\ \hline
      3 & 2,3時間\rule[-3mm]{0mm}{8mm} \\ \hline
      4 & 0.5時間\rule[-3mm]{0mm}{8mm} \\ \hline
      5 & 最初の15~30分のスタートダッシュの部分が一気に高速化できそうだなと感じました!\rule[-3mm]{0mm}{8mm} \\ \hline
      6 & 2時間くらい\rule[-3mm]{0mm}{8mm} \\ \hline
      7 & 16分程\rule[-3mm]{0mm}{8mm} \\ \hline
    \end{tabular}
  \end{center}
\end{table}

\begin{table}[htbp]
  \begin{center}
    \caption{自由記述感想}
    \begin{tabular}{cp{30em}}
      \hline
      被験者番号 & 回答\rule[-3mm]{0mm}{8mm} \\ \hline \hline
      1 & 僕の場合、半日くらいたっても全然メロディーが思いつかなくて時間を無駄にしてしまうことがあるのですが、このシステムで作られたメロディーからインスピレーションを受ければ10分前後でいいメロディーが作れるのでは?と思いました!\rule[-3mm]{0mm}{8mm} \\ \hline
      2 & 全作曲時間の20%程だが、時間短縮と言うよりは曲の数が増えるイメージ\rule[-3mm]{0mm}{8mm} \\ \hline
      3 & 2,3時間\rule[-3mm]{0mm}{8mm} \\ \hline
      4 & 0.5時間\rule[-3mm]{0mm}{8mm} \\ \hline
      5 & 最初の15~30分のスタートダッシュの部分が一気に高速化できそうだなと感じました!\rule[-3mm]{0mm}{8mm} \\ \hline
      6 & 2時間くらい\rule[-3mm]{0mm}{8mm} \\ \hline
      7 & 16分程\rule[-3mm]{0mm}{8mm} \\ \hline
    \end{tabular}
  \end{center}
\end{table}