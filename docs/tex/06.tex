\chapter{考察}
\section{本システムの有用性と満足度}
表5.3の結果より、過半数の人が独自性が高いと回答したことから、本システムが生成するメロディの独自性は高いといえる。しかし、その他と回答した人のなかで、「メロディにオリジナル性は感じたが、ランダムで生成されたメロディであると感じることが多かった」という意見があった。以上のことから、独自性は高いが、Progressive Houseらしさに欠けていたと考えられる。表5.4、5.5の結果より、本システムが生成したメロディによってメロディアイデアが思い浮かび、作曲意欲が向上したといえる。表5.7の結果より、本システムが生成する独自性の高いメロディからインスピレーションを受け、作曲時間の短縮が期待できるといえる。
すべて否定的な回答をするひとはおらず、表5.6の結果より、役に立ちそうかという質問に「はい」と回答した人の割合が100\%であることから、本研究で構築したシステムの有用性が示せたといえる。
\section{今後の課題}
本システムが生成するメロディは、独自性は高いがProgressive houseらしさに欠けるといった問題点が挙げられる。今後の課題として、ペンタトニックスケールを意識し、より調性感\cite{Hoshino84}を感じるメロディを生成することが挙げられる。現状は初期世代メロディのをランダムに生成しているが、学習データの音高の変化に基づいて生成することで,調性感の向上が可能となると考えられる.また、「コード進行を複数の種類から選べられるようにしてほしい」といった意見もあった。現状は、一般的に王道進行と呼ばれる4536進行のみだが、複数のコード進行のなかから選択可能にすることでメロディとコード進行との相性を確認でき、ユーザの満足度を高めることができると考えられる。
%TODO 参考文献
