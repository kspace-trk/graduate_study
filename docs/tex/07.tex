\chapter{おわりに}
本研究では、Progressive Houseを対象としたIGAによるメロディ生成支援システムを構築した。評価実験の結果より、本システムの有用性に対する質問に「はい」と回答した人が100\%であったため本システムの有用性が示せたといえる。しかし、ランダムに生成されたと感じやすいメロディであることがわかった。学習データの音高の変化に基づいて生成し調性感を向上させることで、よりProgressive Houseらしいメロディが生成できると考えられる。また、複数のコード進行のなかから選択できるようにすることで、ユーザの満足度を高めることができると考えられる。
